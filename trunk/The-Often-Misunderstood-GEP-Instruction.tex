\documentclass[12pt,a4paper]{article}
\usepackage{fontspec}
\usepackage{graphicx}
\usepackage{xeCJK}
\setmainfont{SimSun}
\title{常误解的GEP指令}
\author{原作者:Reid Spencer\\译者:张祖羽@HEU\\译文版本:0.11}

\begin{document}
\maketitle

\section{简介}

该文档用于解决围绕LLVM GetElementPtr(GEP)指令的迷惑。关于wiley GEP指令的问题可能是开发者进行LLVM编程时最常遇到的问题。这里我们展示迷惑的源码,显示出GEP指令相当简单。

\section{问题}

当人们首次遇到GEP指令时,他们试图将其与其它编程范例的熟知概念(C数组索引和域选取)联系起来。然而GEP稍有差异,且这导致以下问题;所有这些问题在下面的部分有解答。

\subsection{什么是GEP指令的首个索引?}

快速解答:首个操作数跨步的索引。

首个索引的迷惑通常是认为GetElementPtr指令就像C索引操作。它们不一样。例如,当我们写"C"代码时:

AType *Foo;
...
X = \&Foo->F;

认为这仅有一个索引(域F的选取)是自然的想法。但是在本例,Foo是个指针。该指针在LLVM必须显式索引。而C则是通过指针透明索引。为像C代码一样到达同一地址位置,需提供带有两个索引操作数的GEP指令。首个操作数索引指针;第二个操作数索引结构的域F,如下:

X = \&Foo[0].F;

有时本问题如下描述:
为什么可以通过首个指针索引,而后继指针却不能间接引用(dereference)?

答案很简单,因为执行计算时没必要访存。GEP指令的首个操作数必须是指针类型的值。直接提供给GEP指令的指针值作为操作数,无需访存。因此该指令需要索引操作数。考虑这个示例:

struct munger\_struct {
  int f1;
  int f2;
};
void munge(struct munger\_struct *P) {
  P[0].f1 = P[1].f1 + P[2].f2;
}
...
munger\_struct Array[3];
...
munge(Array);

在每个示例中,首个操作数是GEP指令起始的指针。无论首个操作数是参数、分配的内存,或是全局变量,都是一样的。

为弄清楚,让我们考虑更简单(obtuse)的示例:

\%MyVar = unintialized global i32
...
\%idx1 = getelementptr i32* \%MyVar, i64 0
\%idx2 = getelementptr i32* \%MyVar, i64 1
\%idx3 = getelementptr i32* \%MyVar, i64 2

这些GEP指令简单地完成基于MyVar基址的地址计算。指令如下计算(使用C语法):

idx1 = (char*) \&MyVar + 0
idx2 = (char*) \&MyVar + 4
idx3 = (char*) \&MyVar + 8

既然i32类型是熟知的4字节长度,索引 0, 1 和 2 分别翻译成内存偏移 0, 4 和 8。完成这些计算时没有访存,是因为 \%MyVar 的地址是直接传递给 GEP 指令的。

该示例的简单部分在于 \%idx2 和 \%idx3 。上述变量产生地址(指向全局值 \%MyVar 尾部内存的地址)计算,计算结果仅有一个i32长度而非3个i32长度。虽然在LLVM这是合法的,但不建议使用,因为利用GEP指令产生的指针的load和store将出现未定义的结果。

\subsection{为什么需要额外的0索引?}

快速解答:这没有多余的索引。

该问题是由GEP指令应用在总是指针类型的全局变量时引起的。例如,考虑这个:

Struct = uninitialized global { float*, i32 }
...
\%idx = getelementptr { float*, i32 }* \%MyStruct, i64 0, i32 1

上述GEP指令产生由 \%MyStruct 结构的 i32 类型域 索引的 i32*。当首次见到此指令时,会疑惑为什么需要 i64 0索引。然而,对全局值和GEP指令怎样工作的进一步认识显示该需求。对下面事实的理解可解决迷惑:

\begin{enumerate}
\item \%MyStruct 的类型是 { float*, i32}* 而非 { float*, i32}。那就是说, \%MyStruct 是指向结构的指针,该结构包含指向float的指针和i32。
\item 第一条由此证明:注意GEP指令首个操作数(\%MyStruct)的类型是 {float*, i32}* 。
\item 需要首个索引(i64 0)以明确全局变量 \%MyStruct 的跨步。既然GEP指令的首个参数必须总是指针类型的值,首个索引通过指针跨步。0值意味着该指针的0元素偏移。
\item 第二个索引, i32 1选择结构的第二个域(i32)。
\end{enumerate}

\subsection{什么可由GEP间接引用?}

快速解答:任何值都不可以。

GetElementPtr指令不能间接引用任何值。那就是说,该指令任何时候都不访存。那是由 load 和 store 指令完成的。GEP指令仅在地址计算时涉及(involve)。例如,考虑这样:

\%MyVar = uninitialized global { [40 x i32 ]* }
...
\%idx = getelementptr { [40 x i32]* }* \%MyVar, i64 0, i32 0, i64 0, i64 17

本例中,全局变量 \%MyVar 是指向结构的指针,该结构包含指向40个整数数组的指针。GEP指令似乎是访问结构的整数数组中第18个整数。然而,这是个完全非法的GEP指令。它不能通过编译。其原因是结构的指针必须间接引用,以索引40个整数的数组。既然GEP指令永不访存,这是非法的。

为了访问数组的第18个整数,需要如下实现:

\%idx = getelementptr { [40 x i32]* }* \%, i64 0, i32 0
\%arr = load [40 x i32]** \%idx
\%idx = getelementptr [40 x i32]* \%arr, i64 0, i64 17

既然这样,不得不在索引数组前,使用 load 指令加载结构的指针。如果示例改变了:

\%MyVar = uninitialized global { [40 x i32 ] }
...
\%idx = getelementptr { [40 x i32] }*, i64 0, i32 0, i64 17

则一切工作地很好。既然这样,不包含指针的结构和GEP指令可通过全局变量,索引结构的首个域,并访问数组的第18个i32。

\subsection{为什么不将 GEP x,0,0,1 别名为 GEP x,1 ?}

快速解答:他们计算不同的地址位置。

如果查看这些GEP指令的首个索引,会发现它们是不同的(0 和 1),因此地址计算因索引而不同。考虑该示例:

\%MyVar = global { [10 x i32 ] }
\%idx1 = getlementptr { [10 x i32 ] }* \%MyVar, i64 0, i32 0, i64 1
\%idx2 = getlementptr { [10 x i32 ] }* \%MyVar, i64 1

该示例中, idx1 计算 \%MyVar 结构的数组第二个整数的地址,其值为 MyVal + 4 。idx1 的类型是 i32* 。然而, idx2计算 \%MyVar 后的下个结构地址。 idx2 的类型是 { [10 x i32] }* ,且其值等价于 MyVal + 40, 因为索引经过 MyVar 的 10 个 4 位整数。显然,这种情况下,指针不能别名。

\subsection{为什么将 GEP x,1,0,0 别名为 GEP x,1 ?}

快速解答:它们计算同一地址位置。

这两个GEP指令将计算同一地址位置,因为通过第0个元素的索引不改变地址。然而,这确实改变类型。考虑该示例:


\%MyVar = global { [10 x i32 ] }
\%idx1 = getlementptr { [10 x i32 ] }* \%MyVar, i64 1, i32 0, i64 0
\%idx2 = getlementptr { [10 x i32 ] }* \%MyVar, i64 1

该示例中, \%idx1的值是 \%MyVar + 40,且其类型是 i32* 。\%idx2 的值也是 MyVal + 40 ,但其类型是 { [10 x i32] }* 。

\section{总结}

总之,这是关于 GetElementPtr 指令需要记住的一些事情:

\begin{enumerate}
\item GEP指令永不访存,它仅提供指针计算。
\item GEP指令的首个操作数总是指针,且必须索引。
\item GEP指令没有多余的索引。
\item 后面的零索引对指针别名是多余的,但对于指针类型并非如此。
\item 首个的零索引对指针别名和指针类型都不是多余的。
\end{enumerate}

\end{document}
